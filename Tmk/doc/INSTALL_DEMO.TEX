\documentstyle[11pt]{article}

\textheight 9in         % 1in top and bottom margin for U.S. letter paper
\textwidth 6.5in        % 1in left and right margin
\oddsidemargin 0in      % Both side margins are now 1in
\evensidemargin 0in
\topmargin -.5in \headsep .5in

\begin{document}

\begin{center}
\Large \bf TreadMarks$^{\rm TM}$ Installation -- Demo Version
\end{center}

This demo version consists of the TreadMarks$^{\rm TM}$ library
compiled for a specific architecture, documentation for the library,
and some example programs that use TreadMarks$^{\rm TM}$.  To make
this demo version function, you will need a demo license key.  To
obtain a demo license key, send a request by e-mail to {\tt
tmk@cs.rice.edu}.  During the period for which the demo license key is
valid, you can run both the example programs and your own programs on
up to eight machines.  When the license expires, the library will
continue to function, but only on a single machine.

\noindent
{\bf Hardware/Software Requirements:}
\begin{itemize}
\item A minimum of 72~MBytes of available swap space on each of your machines
\item A file system that is shared by all of your machines and mounted
at the {\em same directory\/} on each one
\item A C or C++ compiler, either gcc version 2.x or the machine
manufacturer's compiler
\end{itemize}

\noindent
{\bf Instructions:}
You have received a uuencoded, compressed tar file.  To install the
TreadMarks$^{\rm TM}$ library and example programs:
\begin{enumerate}
\item
Uudecode, uncompress, and untar the TreadMarks$^{\rm TM}$ distribution
file on a file system shared by all of the machines on which you
intend to run TreadMarks$^{\rm TM}$ programs.
This will create a directory called ``Tmk-i.j.k.demo'', where
``i.j.k'' is the version number.
There are subdirectories for the library, for the example programs,
for the documentation, and for the TreadMarks$^{\rm TM}$ header files.

\item
If your shell is a {\tt csh}-variant, define the environment variable
{\tt TMK\_DEMO\_KEY} in your {\tt .cshrc} file.  Warning: {\em defining {\tt
TMK\_DEMO\_KEY} in {\tt .login} won't work!}  If you use a different
shell, e.g., {\tt bash}, define {\tt TMK\_DEMO\_KEY} in the equivalent
file, e.g., {\tt .bashrc}.  Don't forget to update the environment
of your current interactive shell.

\item
Verify that {\tt rsh\/} can start a process on each machine that you
intend to use.  A good test is {\tt rsh machine echo '\$\{TMK\_DEMO\_KEY\}'}.
If it succeeds, it will display the demo license key.

\end{enumerate}
At this point, you are ready to run the example programs.
You can ``cd'' into the platform-specific subdirectory (e.g.,
``Tmk-i.j.k.demo/apps/tsp\_c/bin.i386\_freebsd2'') and type the name of
the executable (e.g.,
``tsp.udp'') followed by ``-- -- --h machine0 --h machine1 ...''.

You must start the programs from a shell on the first machine
listed on the command line.  Otherwise, an error message will be
printed and the program will terminate.

In addition, you can ``make'' the example programs after changing them
or experiment with programs of your own design.
For more details, please refer to the documentation
in the ``Tmk-i.j.k.demo/doc'' directory.

\vfill

\noindent
Copyright \copyright 1998 by ParallelTools, L.L.C.

\noindent
All rights reserved.

\end{document}
